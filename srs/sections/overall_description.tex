\section{Overall Description}

\subsection{Product Perspective}

"11th Lesson" is a website that organizes all the reading materials a student needs before exam night and minimizes the time for preparation. The system leverages proper guideline for the learners and tutors to learn and showcase their academic knowledge.

\subsection{Product Features}

User Authentication: Secures the system and ensures only authorized users can access sensitive data.

Data Management: Facilitates storage, retrieval, and updating of vehicle documents.

Role-Based Access Control: Differentiates access levels based on user roles (e.g., admin, user).

Responsive Design: Ensures accessibility across various devices, including smartphones and tablets.

\subsection{User Classes and Characteristics}

Students: Primary users of the platform, ranging from high school to college-level students.

Usage Pattern: Likely to use the platform intensively during exam periods, often engaging in self-paced learning sessions.

Expectations: Easy access to concise, high-quality study materials, interactive tools for practice, and immediate feedback on quizzes.

Teachers: Educators responsible for creating, managing, and delivering content on the platform.

Usage Pattern: Regular usage for content management, lesson planning, and student engagement; peak usage before and during exam seasons.

Expectations: Tools for easy content creation, efficient student progress tracking, and features that allow for real-time interaction with students.

Administrators: Users responsible for managing the overall operation of the platform, including user accounts, content approval, and system settings.

Usage Pattern: Consistent usage throughout the year with varying intensity depending on platform updates, user support needs, and content management activities.

Expectations: Robust administrative tools for user management, detailed analytics, system monitoring, and an intuitive interface for overseeing platform operations.

\subsection{Operating Environment}

Hardware: Smartphones or tablets with basic browsing capabilities. Also, computers for administrators to access the web-based dashboard.

Software: Web application accessible via modern browsers (e.g., Chrome, Firefox, Safari). And backend servers for data processing and storage.

Network: Reliable internet connectivity for real-time data access and synchronization.

And secure communication protocols (e.g., HTTPS) for data transmission.

Database: A robust database system (e.g., MySQL) to store vehicle documents and user information.

\subsection{Design and Implementation Constraints}

Open source: A totally free platform to access quality academic study materials.

Scalability: System should handle a large number of simultaneous users and data entries.

Performance: Real-time verification requires efficient data processing and minimal latency.

Compatibility: Must be compatible with a wide range of devices and operating systems.

\subsection{User Documentation}

User Manuals: Comprehensive guides for different user roles detailing how to use the system.

Quick Start Guides: Short tutorials for new users to get started quickly.

Online Help: Contextual help available within the application.

FAQs: Frequently asked questions to address common user queries.

Support Portal: Platform for users to submit support tickets and track their status.

\subsection{Assumptions and Dependencies}

Assumptions: Reliable internet connectivity is available for system operations. Existing membership databases are accessible for integration.

Dependencies: External authentication services (e.g., OAuth providers) if used. Hosting services for deploying the backend and web applications.
